\documentclass{mproj}
\usepackage{graphicx}

\usepackage[utf8]{inputenc}
\usepackage{url}
\usepackage{fancyvrb}
\usepackage[final]{pdfpages}
\usepackage{times}
\usepackage{todonotes}
\usepackage{titlesec}
\usepackage{enumitem}
\setlist{nolistsep}

% Links and their colors
\usepackage[
  colorlinks=true,
  linkcolor=darkgray,
  citecolor=darkgray,
  urlcolor=darkgray,
  ]{hyperref}

% Use a single line for chapter headers
\titleformat{\chapter}[hang]{\normalfont\huge\bfseries}{\thechapter.}{1em}{} 
% Remove the space before chapter titles
\titlespacing*{\chapter}{0pt}{0pt}{40pt}

% Appendices
\usepackage[header,title,titletoc]{appendix}
\renewcommand{\appendixname}{Appendix}

% License
\usepackage[
    type={CC},
    modifier={by-sa},
    version={3.0},
]{doclicense}

% Add bibliography to TOC
\usepackage[nottoc,numbib]{tocbibind}

% for alternative page numbering use the following package
% and see documentation for commands
%\usepackage{fancyheadings}


% other potentially useful packages
%\uspackage{amssymb,amsmath}
%\usepackage{url}
%\usepackage{fancyvrb}
%\usepackage[final]{pdfpages}

\begin{document}

%%%%%%%%%%%%%%%%%%%%%%%%%%%%%%%%%%%%%%%%%%%%%%%%%%%%%%%%%%%%%%%%%%%
\title{$\pi$-calculus session types in Coq \\
\large Using a parametric higher order abstract syntax with dependent types}
\author{Uma Zalakain}
\date{2019-09-06}
\maketitle
%%%%%%%%%%%%%%%%%%%%%%%%%%%%%%%%%%%%%%%%%%%%%%%%%%%%%%%%%%%%%%%%%%%

%%%%%%%%%%%%%%%%%%%%%%%%%%%%%%%%%%%%%%%%%%%%%%%%%%%%%%%%%%%%%%%%%%%
\begin{abstract}
abstract goes here
\end{abstract}
%%%%%%%%%%%%%%%%%%%%%%%%%%%%%%%%%%%%%%%%%%%%%%%%%%%%%%%%%%%%%%%%%%%

%%%%%%%%%%%%%%%%%%%%%%%%%%%%%%%%%%%%%%%%%%%%%%%%%%%%%%%%%%%%%%%%%%%
\educationalconsent
\vfill{}
\doclicenseThis
\newpage
%%%%%%%%%%%%%%%%%%%%%%%%%%%%%%%%%%%%%%%%%%%%%%%%%%%%%%%%%%%%%%%%%%%

%%%%%%%%%%%%%%%%%%%%%%%%%%%%%%%%%%%%%%%%%%%%%%%%%%%%%%%%%%%%%%%%%%%
\section*{Acknowledgements}

acknowledgements go here

%%%%%%%%%%%%%%%%%%%%%%%%%%%%%%%%%%%%%%%%%%%%%%%%%%%%%%%%%%%%%%%%%%%
\tableofcontents
%%%%%%%%%%%%%%%%%%%%%%%%%%%%%%%%%%%%%%%%%%%%%%%%%%%%%%%%%%%%%%%%%%%

%%%%%%%%%%%%%%%%%%%%%%%%%%%%%%%%%%%%%%%%%%%%%%%%%%%%%%%%%%%%%%%%%%%
\chapter{Introduction}\label{intro}

The pi calculus models the exchange of messages between processes over
two-endpoint communication channels. When two processes exchange a message
through a channel computation advances: the receiving process gets reduced
through substitution. Channels themselves can also be sent as messages, and then
be used for further communication.

It is often desirable to restrict the processes that are expressible to those
that are deemed meaningful, leaving out the ones with undesirable features. This
is usually accomplished by adding a type system to the calculus. In the pi
calculus, session types can be used to impose types onto channel endpoints --
the session type of a channel endpoint is a finite sequence of types, each with
an associated direction.

The two endpoints of a channel must have dual session types: if one end is
sending data of type \texttt{T}, the other must be receiving data of type
\texttt{T} -- ensuring \textit{communication safety}. Channel endpoints can be
passed along as messages, but can never be duplicated: communication must only
ever happen between two processes -- ensuring \textit{privacy}.  Processes must
follow the session types of channel endpoints: channels must be used as per
their specification -- ensuring \textit{session fidelity}. As communication
occurs and messages are exchanged, the session types of channels advance.
Ensuring that this reduction process respects the session types involves proving
both \textit{subject reduction} and \textit{type soundness}. \cite{Dardha2016m}

Using a proof assistant to prove these theorems provides a strong guarantee of
the correctness of the proofs. Moreover, it means that any process described in
these terms will have these theorems automatically derived. From the numerous
proof assistants available, this project uses Coq to formalise the pi calculus
with session types. Amongst its features, Coq includes a powerful tactic engine
and a dependent type system, both key for the development of this project.

Past efforts in formalising session types in Coq have created an object language
and then handled variable references, typing contexts, and typing judgments
manually \cite{Dilmore2019}. The present work lifts those responsibilities to
the metalanguage by using a parametric higher order abstract syntax to model
processes. This results in the following key features:

\begin{itemize}
    \item \textbf{Variable references are handled by Coq}: no substitution
        related theorems are required, references to both messages and channels
        feel natural.
    \item \textbf{The channel type is parametrised}, ensuring that channels can
        only be created as defined by the calculus -- no cheating is allowed.
    \item Congruence and reduction are defined inductively on processes;
        reduction entails pattern matching against functions. \textbf{Tactics
        are used to find congruent processes that reduce}.
    \item With the exception of linearity, \textbf{processes are correct by
        construction}: subject reduction of session types is driven by the
        \textbf{continuation passing} of channels. Type preservation is
        therefore defined as: if $P$ is linear and $P$ reduces to $Q$, then $Q$
        is linear.
\end{itemize}


%%%%%%%%%%%%%%%%%%%%%%%%%%%%%%%%%%%%%%%%%%%%%%%%%%%%%%%%%%%%%%%%%%%
% it is fine to change the bibliography style if you want
\bibliographystyle{plain}
\bibliography{mproj}
\end{document}
