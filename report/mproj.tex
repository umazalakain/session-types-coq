\documentclass{mproj}
\usepackage{graphicx}

\usepackage[utf8]{inputenc}
\usepackage{url}
\usepackage{fancyvrb}
\usepackage[final]{pdfpages}
\usepackage{times}
\usepackage{todonotes}
\usepackage{titlesec}
\usepackage{enumitem}
\setlist{nolistsep}

% Links and their colors
\usepackage[
  colorlinks=true,
  linkcolor=darkgray,
  citecolor=darkgray,
  urlcolor=darkgray,
  ]{hyperref}

% Use a single line for chapter headers
\titleformat{\chapter}[hang]{\normalfont\huge\bfseries}{\thechapter.}{1em}{} 
% Remove the space before chapter titles
\titlespacing*{\chapter}{0pt}{0pt}{40pt}

% Appendices
\usepackage[header,title,titletoc]{appendix}
\renewcommand{\appendixname}{Appendix}

% License
\usepackage[
    type={CC},
    modifier={by-sa},
    version={3.0},
]{doclicense}

% Add bibliography to TOC
\usepackage[nottoc,numbib]{tocbibind}

% for alternative page numbering use the following package
% and see documentation for commands
%\usepackage{fancyheadings}


% other potentially useful packages
%\uspackage{amssymb,amsmath}
%\usepackage{url}
%\usepackage{fancyvrb}
%\usepackage[final]{pdfpages}

\begin{document}

%%%%%%%%%%%%%%%%%%%%%%%%%%%%%%%%%%%%%%%%%%%%%%%%%%%%%%%%%%%%%%%%%%%
\title{$\pi$-calculus session types in Coq \\
\large Using a parametric higher order abstract syntax with dependent types}
\author{Uma Zalakain}
\date{2019-09-06}
\maketitle
%%%%%%%%%%%%%%%%%%%%%%%%%%%%%%%%%%%%%%%%%%%%%%%%%%%%%%%%%%%%%%%%%%%

%%%%%%%%%%%%%%%%%%%%%%%%%%%%%%%%%%%%%%%%%%%%%%%%%%%%%%%%%%%%%%%%%%%
\begin{abstract}
abstract goes here
\end{abstract}
%%%%%%%%%%%%%%%%%%%%%%%%%%%%%%%%%%%%%%%%%%%%%%%%%%%%%%%%%%%%%%%%%%%

%%%%%%%%%%%%%%%%%%%%%%%%%%%%%%%%%%%%%%%%%%%%%%%%%%%%%%%%%%%%%%%%%%%
\educationalconsent
\vfill{}
\doclicenseThis
\newpage
%%%%%%%%%%%%%%%%%%%%%%%%%%%%%%%%%%%%%%%%%%%%%%%%%%%%%%%%%%%%%%%%%%%

%%%%%%%%%%%%%%%%%%%%%%%%%%%%%%%%%%%%%%%%%%%%%%%%%%%%%%%%%%%%%%%%%%%
\section*{Acknowledgements}

acknowledgements go here

%%%%%%%%%%%%%%%%%%%%%%%%%%%%%%%%%%%%%%%%%%%%%%%%%%%%%%%%%%%%%%%%%%%
\tableofcontents
%%%%%%%%%%%%%%%%%%%%%%%%%%%%%%%%%%%%%%%%%%%%%%%%%%%%%%%%%%%%%%%%%%%

%%%%%%%%%%%%%%%%%%%%%%%%%%%%%%%%%%%%%%%%%%%%%%%%%%%%%%%%%%%%%%%%%%%
\chapter{Introduction}\label{intro}

The pi calculus models the exchange of messages between processes over
two-endpoint communication channels. Computation advances when two processes
exchange a message through a channel: the receiving process gets reduced through
substitution. Channels themselves can also be sent as messages, and then be used
for further communication.

It is often desirable to restrict the processes that can be expressed to those
that are deemed meaningful, leaving out the ones with undesirable features.
This is usually accomplished by adding a type system to the calculus. In the pi
calculus, session types can be used to impose types onto communication channels
-- the session type of a channel is a finite sequence of types, each with an
associated direction.

The two endpoints of a channel must have dual session types: if one end is
sending data of type \texttt{T}, the other must be receiving data of type
\texttt{T} -- ensuring \textit{communication safety}. The endpoints of a channel
can be passed along as messages, but can never be duplicated: communication can
only ever happen between two processes -- ensuring \textit{privacy}.  Processes
must follow the session types of channel endpoints: channels must be used as per
their specification -- ensuring \textit{session fidelity}. As communication
occurs and messages are exchanged, the session types of channels advance.
Ensuring that this reduction process follows the session types involves proving
both \textit{subject reduction} and \textit{type soundness}.
\cite{Dardha2016m}

Using a proof assistant to prove these theorems provides a strong guarantee of
the correctness of the proofs. Moreover, it means that any process described in
these terms will have these theorems automatically derived. From the numerous
proof assistants available, Coq has been chosen to formalise the pi calculus
with session types. Amongst its features, Coq includes a powerful tactic engine
and a dependent type system, both key for the development of this project.

Past efforts in formalising session types in Coq have created an object language
and then handled variable references, typing contexts, and typing judgments
themselves \cite{Dilmore2019}. The present work lifts those responsibilities to
the metalanguage by using a parametric higher order abstract syntax to model
processes. This results in the following key features:

\begin{itemize}
    \item Variable references are handled by Coq: no substitution related
        theorems are required, references to both messages and channels feel
        natural.
    \item Parametrising the channel type guarantees that channels are only
        created as defined by the calculus -- no cheating is allowed.
    \item The subject reduction of session types is driven by continuation
        passing of channels. Guaranteeing linearity is therefore essential.
    \item Congruence and reduction are relations defined inductively on
        processes; reduction entails pattern matching against functions.
    \item Tactics are used to find congruent processes that reduce.
    \item With the exception of linearity, processes are correct by
        construction. Type safety is therefore defined as: if $P$ is linear and
        $P$ reduces to $Q$, then $Q$ is linear.
\end{itemize}


%%%%%%%%%%%%%%%%%%%%%%%%%%%%%%%%%%%%%%%%%%%%%%%%%%%%%%%%%%%%%%%%%%%
% it is fine to change the bibliography style if you want
\bibliographystyle{plain}
\bibliography{mproj}
\end{document}
