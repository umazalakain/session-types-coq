\documentclass{mproj}
\usepackage{graphicx}

\usepackage{amsmath}
\usepackage[utf8]{inputenc}
\usepackage{url}
\usepackage{fancyvrb}
\usepackage[final]{pdfpages}
\usepackage{times}
\usepackage{todonotes}
\usepackage{titlesec}
\usepackage{enumitem}
\setlist{nolistsep}

% Inference rules
\usepackage{mathpartir}

% Links and their colors
\usepackage[
  colorlinks=true,
  linkcolor=darkgray,
  citecolor=darkgray,
  urlcolor=darkgray,
  ]{hyperref}

% Use a single line for chapter headers
\titleformat{\chapter}[hang]{\normalfont\huge\bfseries}{\thechapter.}{1em}{} 
% Remove the space before chapter titles
\titlespacing*{\chapter}{0pt}{0pt}{40pt}

% Appendices
\usepackage[header,title,titletoc]{appendix}
\renewcommand{\appendixname}{Appendix}

% License
\usepackage[
    type={CC},
    modifier={by-sa},
    version={3.0},
]{doclicense}

% Add bibliography to TOC
\usepackage[nottoc,numbib]{tocbibind}

\newtheorem{theorem}{Theorem}

% Commands for the pi-calculus
\newcommand{\PO}{\mathbf{0}}
\newcommand{\comp}[2]{#1 \mid #2}
\newcommand{\new}[2]{(\boldsymbol{\nu} #1 #2)}
\newcommand{\cout}[2]{\overline{#1}\langle#2\rangle.}
\newcommand{\cin}[2]{#1(#2)}
\newcommand{\select}[2]{#1\triangleleft#2.}
\newcommand{\branch}[2]{#1\triangleright#2}

\newcommand{\subst}[3]{#1[#2/#3]}

\def\picalc/{\(\pi\)-calculus}
\def\Picalc/{\(\pi\)-Calculus}

\newcommand{\type}{\texttt}
\newcommand{\End}{\type{End}}
\newcommand{\Send}[1]{!#1.}
\newcommand{\Recv}[1]{?#1.}
\newcommand{\Select}{\oplus}
\newcommand{\Branch}{\&}
\newcommand{\dual}{\overline}

\newcommand{\reduce}{\rightarrow}
\renewcommand{\emptyset}{\varnothing}
\newcommand{\types}{\vdash}
\begin{document}

%%%%%%%%%%%%%%%%%%%%%%%%%%%%%%%%%%%%%%%%%%%%%%%%%%%%%%%%%%%%%%%%%%%
\title{Type-checking\\ $\pi$-calculus session types\\ with Coq}
\author{Uma Zalakain}
\date{2019-09-06}
\maketitle
%%%%%%%%%%%%%%%%%%%%%%%%%%%%%%%%%%%%%%%%%%%%%%%%%%%%%%%%%%%%%%%%%%%

%%%%%%%%%%%%%%%%%%%%%%%%%%%%%%%%%%%%%%%%%%%%%%%%%%%%%%%%%%%%%%%%%%%
\begin{abstract}
abstract goes here
\end{abstract}
%%%%%%%%%%%%%%%%%%%%%%%%%%%%%%%%%%%%%%%%%%%%%%%%%%%%%%%%%%%%%%%%%%%

%%%%%%%%%%%%%%%%%%%%%%%%%%%%%%%%%%%%%%%%%%%%%%%%%%%%%%%%%%%%%%%%%%%
\educationalconsent
\vfill{}
\doclicenseThis
\newpage
%%%%%%%%%%%%%%%%%%%%%%%%%%%%%%%%%%%%%%%%%%%%%%%%%%%%%%%%%%%%%%%%%%%

%%%%%%%%%%%%%%%%%%%%%%%%%%%%%%%%%%%%%%%%%%%%%%%%%%%%%%%%%%%%%%%%%%%
\section*{Acknowledgements}

Acknowledgements go here

%%%%%%%%%%%%%%%%%%%%%%%%%%%%%%%%%%%%%%%%%%%%%%%%%%%%%%%%%%%%%%%%%%%
\tableofcontents
%%%%%%%%%%%%%%%%%%%%%%%%%%%%%%%%%%%%%%%%%%%%%%%%%%%%%%%%%%%%%%%%%%%

%%%%%%%%%%%%%%%%%%%%%%%%%%%%%%%%%%%%%%%%%%%%%%%%%%%%%%%%%%%%%%%%%%%
\chapter{Introduction}\label{intro}
%%%%%%%%%%%%%%%%%%%%%%%%%%%%%%%%%%%%%%%%%%%%%%%%%%%%%%%%%%%%%%%%%%%

The pi calculus models the exchange of messages between processes over
two-endpoint communication channels. When two processes exchange a message over
a channel computation advances: the receiving process gets reduced through
substitution; the sending process proceeds further. Channels have types,
limiting the data that can be exchanged over them. A brief introduction to the
pi calculus is provided in \S \ref{pi-calculus}.

Session types add sequential types to channels: the session type of a channel
endpoint is a finite sequence of types, each with an associated direction.  When
a channel endpoint is used by a process, the type and direction of the exchanged
data must follow the specification of the associated session type. The theorems
enabled by this type system are found in \S \ref{session-types}.

Using a proof assistant to prove these theorems provides an exceptionally strong
guarantee of their correctness. Moreover, it means that any kind of process that
uses session types as an abstraction can have these theorems automatically
derived for it. From the numerous proof assistants available, this project uses
Coq to formalise the pi calculus with session types. Amongst its features, Coq
includes a powerful tactic engine and a dependent type system, both key for the
development of this project. An overview of Coq is given in \S \ref{coq}.

The present work takes advantage of polymorphism (\S \ref{polymorphism}) and
dependent types (\S \ref{dependent-types}) and uses a parametric higher-order
abstract syntax (\S \ref{phoas}) and continuation passing (\S
\ref{continuation-passing}) to \textbf{discharge the type-checking of pi
calculus terms on Coq}. However, this method relies on channel endpoints being
used exactly once, i.e. linearly (\S \ref{linearity}). Coq does not support
linearity, and thus this check will have to be addressed ad-hoc. Once a
linearity check is available, the type-checking of terms is decidable and
\textbf{variable references, typing contexts and typing judgments can be lifted
to the host language}. The details of how this is carried out in practice are
explained in \S \ref{implementation}.

Past efforts in formalising session types in Coq have created object languages
and handled variable references, typing contexts, and typing judgments by hand
\cite{Dilmore2019}. These and other approaches are mentioned in \S
\ref{related-work}.

Finally, \S \ref{conclusion} suggests future work that might be of interest, and
offers conclusions on what this project has achieved.

%%%%%%%%%%%%%%%%%%%%%%%%%%%%%%%%%%%%%%%%%%%%%%%%%%%%%%%%%%%%%%%%%%%
\chapter{Background knowledge}
%%%%%%%%%%%%%%%%%%%%%%%%%%%%%%%%%%%%%%%%%%%%%%%%%%%%%%%%%%%%%%%%%%%

%%%%%%%%%%%%%%%%%%%%%%%%%%%%%%%%%%%%%%%%%%%%%%%%%%%%%%%%%%%%%%%%%%%
\section{Pi calculus}\label{pi-calculus}
%%%%%%%%%%%%%%%%%%%%%%%%%%%%%%%%%%%%%%%%%%%%%%%%%%%%%%%%%%%%%%%%%%%

\cite{Vasconcelos2009}
\todo{Typing rules}
\todo{Channels as messages}

\begin{align*}
P,Q ::= \; &\PO                                 & \text{inaction}             \\
           &\new{x}{y}P                         & \text{scope restriction}    \\
           &\cout{x}{u}P                        & \text{output}               \\
           &\cin{y}{u}P                         & \text{input}                \\
           &\select{x}{l_j}P                    & \text{selection}            \\
           &\branch{x}{\{l_i : P_i\}_{i \in I}} & \text{branching}            \\
           &\comp{P}{Q}                         & \text{parallel composition}
\end{align*}

\begin{mathpar}
\inferrule
    { }
    {\comp{P}{Q} \equiv \comp{Q}{P}}
    \quad (\textsc{C-CompComm})

\inferrule
    { }
    {\new{x}{y} \new{z}{w} P \equiv \new{z}{w} \new{x}{y} P}
    \quad (\textsc{C-ScopeComm})

\inferrule
    { }
    {\comp{P}{\PO} \equiv P}
    \quad (\textsc{C-Comp0})

\inferrule
    { }
    {\comp {\comp{P}{Q}} {R} \equiv \comp {P} {\comp{Q}{R}}}
    \quad (\textsc{C-CompAssoc})

\inferrule
    { }
    {\new{x}{y} \PO \equiv \PO}
    \quad (\textsc{C-Scope0})

\inferrule
    {x,y \not\in fn(Q)}
    {\comp {\new{x}{y}P} {Q} \equiv \new{x}{y} \comp{P}{Q}}
    \quad (\textsc{C-ScopeExp})

\inferrule
    { }
    {\new{x}{y}P \equiv \new{y}{x}P}
    \quad (\textsc{C-ScopeSwap})
\end{mathpar}

\begin{mathpar}
\inferrule 
    { }
    {\new{x}{y}(\comp {\cout{x}{a}P} {\cin{y}{b}Q}) \reduce
     \new{x}{y}(\comp {P}            {\subst{Q}{a}{b}})}
    \quad (\textsc{R-Comm})

\inferrule
    {j \in I}
    {\new{x}{y}(\comp {\select{x}{l_j}P} {\branch{y}{\{l_i : Q_i\}_{i \in I}}}) \reduce
     \new{x}{y}(\comp {P}                {Q_j})}
    \quad (\textsc{R-Case})

\inferrule
    {P \reduce Q}
    {\new{x}{y}P \reduce \new{x}{y}Q}
    \quad (\textsc{R-Res})

\inferrule
    {P \reduce Q}
    {\comp{P}{R} \reduce \comp{Q}{R}}
    \quad (\textsc{R-Par})

\inferrule
    {P \equiv P' \\ P' \reduce Q' \\ Q' \equiv Q}
    {P \reduce Q}
    \quad (\textsc{R-Struct})
\end{mathpar}

%%%%%%%%%%%%%%%%%%%%%%%%%%%%%%%%%%%%%%%%%%%%%%%%%%%%%%%%%%%%%%%%%%%
\section{Session types}\label{session-types}
%%%%%%%%%%%%%%%%%%%%%%%%%%%%%%%%%%%%%%%%%%%%%%%%%%%%%%%%%%%%%%%%%%%

\todo{Typing rules}
\todo{Channels as messages}
\todo{Guarantees}

The two endpoints of a channel must have dual session types: if one end is
sending data of type \texttt{T}, the other must be receiving data of type
\texttt{T} -- ensuring \textit{communication safety}. Channel endpoints can be
passed along as messages, but can never be duplicated: communication must only
ever occur between two processes -- ensuring \textit{privacy}.  Processes must
follow the session types of channel endpoints: channels must be used as per
their specification -- ensuring \textit{session fidelity}. As communication
occurs and messages are exchanged, the session types of channels advance.
Ensuring that this reduction process respects session types involves proving
both \textit{subject reduction} and \textit{type soundness}. \cite{Dardha2016m}

\begin{mathpar}
\inferrule{}{\dual{\Send{T}S} = \Recv{T}\dual{S}}

\inferrule{}{\dual{\Recv{T}S} = \Send{T}\dual{S}}

\inferrule{}{
    \dual{\Branch \{l_i : S_i\}_{i \in I}} =
    \Select \{l_i : \dual{S_i}\}_{i \in I}}

\inferrule{}{
    \dual{\Select\{l_i : S_i\}_{i \in I}} =
    \Branch \{l_i : \dual{S_i}\}_{i \in I}}

\inferrule{}{\dual{\End} = \End}
\end{mathpar}

\begin{mathpar}
\inferrule
    {\Gamma \types x : \End}
    {\Gamma \types \PO}
    \quad (\textsc{T-Inact})

\inferrule
    {\Gamma_1 \types P \\
     \Gamma_2 \types Q}
    {\Gamma_1 \circ \Gamma_2 \types \comp{P}{Q}}
    \quad (\textsc{T-Par})

\inferrule
    {\Gamma,x:T,y:\dual{T} \types P}
    {\Gamma \types \new{x}{y}P}
    \quad (\textsc{T-Res})

\inferrule
    {\Gamma_1 \types x:\Recv{T}S \\
     \Gamma_2,x:S,y:T \types P}
    {\Gamma_1 \circ \Gamma_2 \types \cin{x}{y}P}
    \quad (\textsc{T-In})

\inferrule
    {\Gamma_1 \types x:\Send{T}S \\
     \Gamma_2 \types v:T \\
     \Gamma_3,x:S \types P}
    {\Gamma_1 \circ \Gamma_2 \circ \Gamma_3 \types \cout{x}{v}P}
    \quad (\textsc{T-Out})

\inferrule
    {\Gamma_1 \types x:\Branch{\{l_i : S_i\}_{i \in I}} \\
     \Gamma_2,x:S_i \types P_i \\
     \forall i \in I}
    {\Gamma_1 \circ \Gamma_2 \types x \branch{\{l_i : P_i\}_{i \in I}}}
    \quad (\textsc{T-Branch})

\inferrule
    {\Gamma_1 \types x:\Select{\{l_i : S_i\}_{i \in I}} \\
     \Gamma_2,x:S_i \types P_i \\
     \exists j \in I}
    {\Gamma_1 \circ \Gamma_2 \types x \select{l_j}P}
    \quad (\textsc{T-Select})
\end{mathpar}

%%%%%%%%%%%%%%%%%%%%%%%%%%%%%%%%%%%%%%%%%%%%%%%%%%%%%%%%%%%%%%%%%%%
\section{The Coq proof assistant}\label{coq}
%%%%%%%%%%%%%%%%%%%%%%%%%%%%%%%%%%%%%%%%%%%%%%%%%%%%%%%%%%%%%%%%%%%

\todo{Powerful tactics}
\todo{Not so good with dependent types}
\todo{Equations package}

%%%%%%%%%%%%%%%%%%%%%%%%%%%%%%%%%%%%%%%%%%%%%%%%%%%%%%%%%%%%%%%%%%%
\section{Polymorphism}\label{polymorphism}
%%%%%%%%%%%%%%%%%%%%%%%%%%%%%%%%%%%%%%%%%%%%%%%%%%%%%%%%%%%%%%%%%%%

\cite{Wadler1989}

%%%%%%%%%%%%%%%%%%%%%%%%%%%%%%%%%%%%%%%%%%%%%%%%%%%%%%%%%%%%%%%%%%%
\section{Dependent types}\label{dependent-types}
%%%%%%%%%%%%%%%%%%%%%%%%%%%%%%%%%%%%%%%%%%%%%%%%%%%%%%%%%%%%%%%%%%%
\todo{Indexed datatypes}

%%%%%%%%%%%%%%%%%%%%%%%%%%%%%%%%%%%%%%%%%%%%%%%%%%%%%%%%%%%%%%%%%%%
\chapter{Encoding}\label{encoding}
%%%%%%%%%%%%%%%%%%%%%%%%%%%%%%%%%%%%%%%%%%%%%%%%%%%%%%%%%%%%%%%%%%%

%%%%%%%%%%%%%%%%%%%%%%%%%%%%%%%%%%%%%%%%%%%%%%%%%%%%%%%%%%%%%%%%%%%
\section{Parametric HOAS}\label{phoas}
%%%%%%%%%%%%%%%%%%%%%%%%%%%%%%%%%%%%%%%%%%%%%%%%%%%%%%%%%%%%%%%%%%%

\cite{Wadler1989}
\cite{Chlipala2008}
\todo{We use polymorphism to make channels opaque}
\todo{We do not do open processes}
\todo{We use polymorphism on messages to make processes traversable}

The introduction of both channels and received messages is modelled as function
abstraction in Coq, therefore \textbf{variables are handled transparently} -- no
substitution related lemmas are required. Channel types are parametrised to make
them opaque -- they cannot be illicitly created or inspected by the user.

%%%%%%%%%%%%%%%%%%%%%%%%%%%%%%%%%%%%%%%%%%%%%%%%%%%%%%%%%%%%%%%%%%%
\section{Continuation passing}\label{continuation-passing}
%%%%%%%%%%%%%%%%%%%%%%%%%%%%%%%%%%%%%%%%%%%%%%%%%%%%%%%%%%%%%%%%%%%

\cite{Vasconcelos2010}
\todo{We merge pi calculus and session types into one}

Assuming linearity, \textbf{processes are correct by construction}: the
processes that can be constructed depend on the session types of the channels in
the environment of the host language; an action strips off the outer layer of a
channel's session type -- modelling \textbf{continuation passing}.

%%%%%%%%%%%%%%%%%%%%%%%%%%%%%%%%%%%%%%%%%%%%%%%%%%%%%%%%%%%%%%%%%%%
\section{Linearity}\label{linearity}
%%%%%%%%%%%%%%%%%%%%%%%%%%%%%%%%%%%%%%%%%%%%%%%%%%%%%%%%%%%%%%%%%%%

\cite{Kobayashi1999}
\cite{Toninho2011}

%%%%%%%%%%%%%%%%%%%%%%%%%%%%%%%%%%%%%%%%%%%%%%%%%%%%%%%%%%%%%%%%%%%
\section{Type preservation}\label{type-preservation}
%%%%%%%%%%%%%%%%%%%%%%%%%%%%%%%%%%%%%%%%%%%%%%%%%%%%%%%%%%%%%%%%%%%

Ensuring that linearity is preserved through reduction is therefore essential:
\begin{theorem}
    $lin(P) \Rightarrow P \rightarrow Q \Rightarrow lin(Q).$
\end{theorem}

%%%%%%%%%%%%%%%%%%%%%%%%%%%%%%%%%%%%%%%%%%%%%%%%%%%%%%%%%%%%%%%%%%%
\chapter{Implementation}\label{implementation}
%%%%%%%%%%%%%%%%%%%%%%%%%%%%%%%%%%%%%%%%%%%%%%%%%%%%%%%%%%%%%%%%%%%

%%%%%%%%%%%%%%%%%%%%%%%%%%%%%%%%%%%%%%%%%%%%%%%%%%%%%%%%%%%%%%%%%%%
\section{Processes}\label{processes}
%%%%%%%%%%%%%%%%%%%%%%%%%%%%%%%%%%%%%%%%%%%%%%%%%%%%%%%%%%%%%%%%%%%
\subsection{Congruence}
\subsection{Reduction}

%%%%%%%%%%%%%%%%%%%%%%%%%%%%%%%%%%%%%%%%%%%%%%%%%%%%%%%%%%%%%%%%%%%
\section{Linearity check}\label{linearity-check}
%%%%%%%%%%%%%%%%%%%%%%%%%%%%%%%%%%%%%%%%%%%%%%%%%%%%%%%%%%%%%%%%%%%

%%%%%%%%%%%%%%%%%%%%%%%%%%%%%%%%%%%%%%%%%%%%%%%%%%%%%%%%%%%%%%%%%%%
\section{Linearity preservation}\label{linearity-preservation}
%%%%%%%%%%%%%%%%%%%%%%%%%%%%%%%%%%%%%%%%%%%%%%%%%%%%%%%%%%%%%%%%%%%

%%%%%%%%%%%%%%%%%%%%%%%%%%%%%%%%%%%%%%%%%%%%%%%%%%%%%%%%%%%%%%%%%%%
\chapter{Related work}\label{related-work}
%%%%%%%%%%%%%%%%%%%%%%%%%%%%%%%%%%%%%%%%%%%%%%%%%%%%%%%%%%%%%%%%%%%

%%%%%%%%%%%%%%%%%%%%%%%%%%%%%%%%%%%%%%%%%%%%%%%%%%%%%%%%%%%%%%%%%%%
\chapter{Conclusion}\label{conclusion}
%%%%%%%%%%%%%%%%%%%%%%%%%%%%%%%%%%%%%%%%%%%%%%%%%%%%%%%%%%%%%%%%%%%

%%%%%%%%%%%%%%%%%%%%%%%%%%%%%%%%%%%%%%%%%%%%%%%%%%%%%%%%%%%%%%%%%%%
\bibliographystyle{plain}
\bibliography{mproj}
\end{document}
