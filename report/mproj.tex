\documentclass{mproj}
\usepackage{graphicx}

\usepackage{amsmath}
\usepackage[utf8]{inputenc}
\usepackage{url}
\usepackage{fancyvrb}
\usepackage[final]{pdfpages}
\usepackage{times}
\usepackage{todonotes}
\usepackage{titlesec}
\usepackage{enumitem}
\setlist{nolistsep}
\usepackage{listings}
% lstlisting coq style (inspired from a file of Assia Mahboubi)

%
\lstdefinelanguage{Coq}{ 
%
% Anything betweeen $ becomes LaTeX math mode
mathescape=true,
%
% Comments may or not include Latex commands
texcl=false, 
%
% Vernacular commands
morekeywords=[1]{Section, Module, End, Require, Import, Export,
  Variable, Variables, Parameter, Parameters, Axiom, Hypothesis,
  Hypotheses, Notation, Local, Tactic, Reserved, Scope, Open, Close,
  Bind, Delimit, Definition, Let, Ltac, Fixpoint, CoFixpoint, Add,
  Morphism, Relation, Implicit, Arguments, Unset, Contextual,
  Strict, Prenex, Implicits, Inductive, CoInductive, Record,
  Structure, Canonical, Coercion, Context, Class, Global, Instance,
  Program, Infix, Theorem, Lemma, Corollary, Proposition, Fact,
  Remark, Example, Proof, Goal, Save, Qed, Defined, Hint, Resolve,
  Rewrite, View, Search, Show, Print, Printing, All, Eval, Check,
  Projections, inside, outside, Def},
%
% Gallina
morekeywords=[2]{forall, exists, exists2, fun, fix, cofix, struct,
  match, with, end, as, in, return, let, if, is, then, else, for, of,
  nosimpl, when},
%
% Sorts
morekeywords=[3]{Type, Prop, Set, true, false, option},
%
% Various tactics, some are std Coq subsumed by ssr, for the manual purpose
morekeywords=[4]{pose, set, move, case, elim, apply, clear, hnf,
  intro, intros, generalize, rename, pattern, after, destruct,
  induction, using, refine, inversion, injection, rewrite, congr,
  unlock, compute, ring, field, fourier, replace, fold, unfold,
  change, cutrewrite, simpl, have, suff, wlog, suffices, without,
  loss, nat_norm, assert, cut, trivial, revert, bool_congr, nat_congr,
  symmetry, transitivity, auto, split, left, right, autorewrite, constructor},
%
% Terminators
morekeywords=[5]{by, done, exact, reflexivity, tauto, romega, omega,
  assumption, solve, contradiction, discriminate},
%
% Control
morekeywords=[6]{do, last, first, try, idtac, repeat},
%
% Comments delimiters, we do turn this off for the manual
morecomment=[s]{(*}{*)},
%
% Spaces are not displayed as a special character
showstringspaces=false,
%
% String delimiters
morestring=[b]",
morestring=[d]’,
%
% Size of tabulations
tabsize=4,
%
% Enables ASCII chars 128 to 255
extendedchars=false,
% Support UTF8
inputencoding=utf8,
%
% Case sensitivity
sensitive=true,
%
% Automatic breaking of long lines
breaklines=false,
%
% Default style fors listings
basicstyle=\small,
%
% Position of captions is bottom
captionpos=b,
%
% flexible columns
columns=fixed,
%
% Style for (listings') identifiers
identifierstyle={\ttfamily\color{black}},
% Style for declaration keywords
keywordstyle=[1]{\ttfamily\color{violet}},
% Style for gallina keywords
keywordstyle=[2]{\ttfamily\color{OliveGreen}},
% Style for sorts keywords
keywordstyle=[3]{\ttfamily\color{ProcessBlue}},
% Style for tactics keywords
keywordstyle=[4]{\ttfamily\color{blue}},
% Style for terminators keywords
keywordstyle=[5]{\ttfamily\color{red}},
%Style for iterators
%keywordstyle=[6]{\ttfamily\color{darkpink}},
% Style for strings
stringstyle=\ttfamily,
% Style for comments
commentstyle={\ttfamily\color{OliveGreen}},
%
%moredelim=**[is][\ttfamily\color{red}]{/&}{&/},
literate=
    {forall}{{\color{OliveGreen}{$\forall\;$}\color{black}}}1
    {exists}{{$\exists\;$}}1
    {<-}{{$\leftarrow\;$}}1
    {=>}{{$\Rightarrow\;$}}1
    {==}{{\code{==}\;}}1
    {==>}{{\code{==>}\;}}1
%    {:>}{{\code{:>}\;}}1
    {->}{{$\rightarrow\;$}}1
    {<->}{{$\leftrightarrow\;$}}1
    {<==}{{$\leq\;$}}1
    {\#}{{$^\star$}}1 
    {\\o}{{$\circ\;$}}1 
    {\@}{{$\cdot$}}1 
    {\/\\}{{$\wedge\;$}}1
    {\\\/}{{$\vee\;$}}1
    {++}{{\code{++}}}1
    {~}{{\ }}1
    {\@\@}{{$@$}}1
    {\\mapsto}{{$\mapsto\;$}}1
    {\\hline}{{\rule{\linewidth}{0.5pt}}}1
    {nat}{{$\mathbb{N}\;$}}1
    {neg}{{$\neg\;$}}1
    {ø}{{$\emptyset\;$}}1
    {⊕}{{$\oplus\;$}}1
    {υ}{{$\varepsilon\;$}}1
    {ε}{{$\epsilon\;$}}1
    {≡}{{$\equiv\;$}}1
    {⇒}{{$\Rightarrow\;$}}1
    {▹}{{$\triangleright\;$}}1
    {◃}{{$\triangleleft\;$}}1
    {fun}{{$\lambda\;$}}1
%
}[keywords,comments,strings]

\lstnewenvironment{coq}{\lstset{language=Coq}}{}

% pour inliner dans le texte
\def\coqe{\lstinline[language=Coq, breaklines=true, basicstyle=\small]}
% pour inliner dans les tableaux / displaymath...
\def\coqes{\lstinline[language=Coq, breaklines=true, basicstyle=\scriptsize]}
% to include from other files
\def\coqi[#1]{\lstinputlisting[language=Coq, basicstyle=\small, linerange=#1]}

%%% Local Variables: 
%%% mode: latex
%%% Local IspellDict: british
%%% TeX-master: "main.tex"
%%% End: 


% Inference rules
\usepackage{mathpartir}

% Links and their colors
\usepackage[
  colorlinks=true,
  linkcolor=darkgray,
  citecolor=darkgray,
  urlcolor=darkgray,
  ]{hyperref}

% Use a single line for chapter headers
\titleformat{\chapter}[hang]{\normalfont\huge\bfseries}{\thechapter.}{1em}{} 
% Remove the space before chapter titles
\titlespacing*{\chapter}{0pt}{0pt}{40pt}

% Appendices
\usepackage[header,title,titletoc]{appendix}
\renewcommand{\appendixname}{Appendix}

% License
\usepackage[
    type={CC},
    modifier={by-sa},
    version={3.0},
]{doclicense}

% Add bibliography to TOC
\usepackage[nottoc,numbib]{tocbibind}

\newtheorem{theorem}{Theorem}

% Commands for the pi-calculus
\newcommand{\PO}{\mathbf{0}}
\newcommand{\comp}[2]{#1 \mid #2}
\newcommand{\new}[2]{(\boldsymbol{\nu} #1 #2)}
\newcommand{\cout}[2]{\overline{#1}\langle#2\rangle.}
\newcommand{\cin}[2]{#1(#2)}
\newcommand{\select}[2]{#1\triangleleft#2.}
\newcommand{\branch}[2]{#1\triangleright#2}

\newcommand{\subst}[3]{#1[#2/#3]}

\def\picalc/{\(\pi\)-calculus}
\def\Picalc/{\(\pi\)-Calculus}

\newcommand{\type}{\texttt}
\newcommand{\End}{\type{End}}
\newcommand{\Send}[1]{!#1.}
\newcommand{\Recv}[1]{?#1.}
\newcommand{\Select}{\oplus}
\newcommand{\Branch}{\&}
\newcommand{\dual}{\overline}

\newcommand{\reduce}{\rightarrow}
\renewcommand{\emptyset}{\varnothing}
\newcommand{\types}{\vdash}
\begin{document}

%%%%%%%%%%%%%%%%%%%%%%%%%%%%%%%%%%%%%%%%%%%%%%%%%%%%%%%%%%%%%%%%%%%
\title{Type-checking\\ $\pi$-calculus session types\\ with Coq}
\author{Uma Zalakain}
\date{2019-09-06}
\maketitle
%%%%%%%%%%%%%%%%%%%%%%%%%%%%%%%%%%%%%%%%%%%%%%%%%%%%%%%%%%%%%%%%%%%

%%%%%%%%%%%%%%%%%%%%%%%%%%%%%%%%%%%%%%%%%%%%%%%%%%%%%%%%%%%%%%%%%%%
\begin{abstract}
abstract goes here
\end{abstract}
%%%%%%%%%%%%%%%%%%%%%%%%%%%%%%%%%%%%%%%%%%%%%%%%%%%%%%%%%%%%%%%%%%%

%%%%%%%%%%%%%%%%%%%%%%%%%%%%%%%%%%%%%%%%%%%%%%%%%%%%%%%%%%%%%%%%%%%
\educationalconsent
\vfill{}
\doclicenseThis
\newpage
%%%%%%%%%%%%%%%%%%%%%%%%%%%%%%%%%%%%%%%%%%%%%%%%%%%%%%%%%%%%%%%%%%%

%%%%%%%%%%%%%%%%%%%%%%%%%%%%%%%%%%%%%%%%%%%%%%%%%%%%%%%%%%%%%%%%%%%
\section*{Acknowledgements}

Acknowledgements go here

%%%%%%%%%%%%%%%%%%%%%%%%%%%%%%%%%%%%%%%%%%%%%%%%%%%%%%%%%%%%%%%%%%%
\tableofcontents
%%%%%%%%%%%%%%%%%%%%%%%%%%%%%%%%%%%%%%%%%%%%%%%%%%%%%%%%%%%%%%%%%%%

%%%%%%%%%%%%%%%%%%%%%%%%%%%%%%%%%%%%%%%%%%%%%%%%%%%%%%%%%%%%%%%%%%%
\chapter{Introduction}\label{intro}
%%%%%%%%%%%%%%%%%%%%%%%%%%%%%%%%%%%%%%%%%%%%%%%%%%%%%%%%%%%%%%%%%%%

The pi calculus models the exchange of messages between agents over two-endpoint
communication channels. When two agents exchange a message over a channel
computation advances: the receiving agent gets reduced through substitution; the
sending agent proceeds further. Channels are typed, limiting the data that can
be exchanged over them. A further introduction to the pi calculus is provided in
\S \ref{pi-calculus}.

Session types add sequential types to channels: the session type of a channel
endpoint is a finite sequence of types, each with an associated direction.  When
a channel endpoint is used by an agent, the type and direction of the exchanged
data must follow the specification of the associated session type. The theorems
enabled by this type system are found in \S \ref{session-types}.

Using a proof assistant to prove these theorems provides an exceptionally strong
guarantee of their correctness. Moreover, it means that any agent that uses
session types as an abstraction can have these theorems automatically derived
for it. From the numerous proof assistants available, this project uses Coq to
formalise the pi calculus with session types. Amongst its features, Coq includes
a powerful tactic engine and a dependent type system, both key for the
development of this project. An overview of Coq is given in \S \ref{coq}.

The present work takes advantage of polymorphism (\S \ref{polymorphism}) and
dependent types (\S \ref{dependent-types}) and uses a parametric higher-order
abstract syntax (\S \ref{phoas}) and continuation passing (\S
\ref{continuation-passing}) to \textbf{discharge the type-checking of pi
calculus terms on Coq}. However, this method relies on channel endpoints being
used exactly once, i.e. linearly (\S \ref{linearity}). Coq does not support
linearity, and thus this check will have to be addressed ad-hoc. Once a
linearity check is available, the type-checking of terms is decidable and
\textbf{variable references, typing contexts and typing judgments can be lifted
to the host language}. The details of how this is carried out in practice are
explained in \S \ref{implementation}.

Past efforts in formalising session types in Coq have created object languages
and handled variable references, typing contexts, and typing judgments by hand
\cite{Dilmore2019}. These and other approaches are mentioned in \S
\ref{related-work}.

Finally, \S \ref{conclusion} suggests future work that might be of interest, and
offers conclusions on what this project has achieved.

%%%%%%%%%%%%%%%%%%%%%%%%%%%%%%%%%%%%%%%%%%%%%%%%%%%%%%%%%%%%%%%%%%%
\chapter{Background knowledge}
%%%%%%%%%%%%%%%%%%%%%%%%%%%%%%%%%%%%%%%%%%%%%%%%%%%%%%%%%%%%%%%%%%%

%%%%%%%%%%%%%%%%%%%%%%%%%%%%%%%%%%%%%%%%%%%%%%%%%%%%%%%%%%%%%%%%%%%
\section{Pi calculus}\label{pi-calculus}
%%%%%%%%%%%%%%%%%%%%%%%%%%%%%%%%%%%%%%%%%%%%%%%%%%%%%%%%%%%%%%%%%%%

\cite{Vasconcelos2009}
\todo{Channels as messages}
\todo{Limit ourselves to binary channels}

\begin{align*}
P,Q ::= \; &\PO                                 & \text{inaction}             \\
           &\new{x}{y}P                         & \text{scope restriction}    \\
           &\cout{x}{u}P                        & \text{output}               \\
           &\cin{y}{u}P                         & \text{input}                \\
           &\select{x}{l_j}P                    & \text{selection}            \\
           &\branch{x}{\{l_i : P_i\}_{i \in I}} & \text{branching}            \\
           &\comp{P}{Q}                         & \text{parallel composition}
\end{align*}

\begin{mathpar}
\inferrule
    { }
    {\comp{P}{Q} \equiv \comp{Q}{P}}
    \quad (\textsc{C-CompComm})

\inferrule
    { }
    {\new{x}{y} \new{z}{w} P \equiv \new{z}{w} \new{x}{y} P}
    \quad (\textsc{C-ScopeComm})

\inferrule
    { }
    {\comp{P}{\PO} \equiv P}
    \quad (\textsc{C-Comp0})

\inferrule
    { }
    {\comp {\comp{P}{Q}} {R} \equiv \comp {P} {\comp{Q}{R}}}
    \quad (\textsc{C-CompAssoc})

\inferrule
    { }
    {\new{x}{y} \PO \equiv \PO}
    \quad (\textsc{C-Scope0})

\inferrule
    {x,y \not\in fn(Q)}
    {\comp {\new{x}{y}P} {Q} \equiv \new{x}{y} \comp{P}{Q}}
    \quad (\textsc{C-ScopeExp})

\inferrule
    { }
    {\new{x}{y}P \equiv \new{y}{x}P}
    \quad (\textsc{C-ScopeSwap})
\end{mathpar}

\begin{mathpar}
\inferrule 
    { }
    {\new{x}{y}(\comp {\cout{x}{a}P} {\cin{y}{b}Q}) \reduce
     \new{x}{y}(\comp {P}            {\subst{Q}{a}{b}})}
    \quad (\textsc{R-Comm})

\inferrule
    {j \in I}
    {\new{x}{y}(\comp {\select{x}{l_j}P} {\branch{y}{\{l_i : Q_i\}_{i \in I}}}) \reduce
     \new{x}{y}(\comp {P}                {Q_j})}
    \quad (\textsc{R-Case})

\inferrule
    {P \reduce Q}
    {\new{x}{y}P \reduce \new{x}{y}Q}
    \quad (\textsc{R-Res})

\inferrule
    {P \reduce Q}
    {\comp{P}{R} \reduce \comp{Q}{R}}
    \quad (\textsc{R-Par})

\inferrule
    {P \equiv P' \\ P' \reduce Q' \\ Q' \equiv Q}
    {P \reduce Q}
    \quad (\textsc{R-Struct})
\end{mathpar}

%%%%%%%%%%%%%%%%%%%%%%%%%%%%%%%%%%%%%%%%%%%%%%%%%%%%%%%%%%%%%%%%%%%
\section{Session types}\label{session-types}
%%%%%%%%%%%%%%%%%%%%%%%%%%%%%%%%%%%%%%%%%%%%%%%%%%%%%%%%%%%%%%%%%%%

\todo{Typing rules}
\todo{Channels as messages}
\todo{Guarantees}
\todo{Limit ourselves to linear types: $\circ$ is union of non intersecting sets}

The two endpoints of a channel must have dual session types: if one end is
sending data of type \texttt{T}, the other must be receiving data of type
\texttt{T} -- ensuring \textit{communication safety}. Channel endpoints can be
passed along as messages, but can never be duplicated: communication must only
ever occur between two agents -- ensuring \textit{privacy}. Agents must
follow the session types of channel endpoints: channels must be used as per
their specification -- ensuring \textit{session fidelity}. As communication
occurs and messages are exchanged, the session types of channels advance.
Ensuring that this reduction process respects session types involves proving
both \textit{subject reduction} and \textit{type soundness}. \cite{Dardha2016m}

\begin{mathpar}
\inferrule{}{\dual{\Send{T}S} = \Recv{T}\dual{S}}

\inferrule{}{\dual{\Recv{T}S} = \Send{T}\dual{S}}

\inferrule{}{
    \dual{\Branch \{l_i : S_i\}_{i \in I}} =
    \Select \{l_i : \dual{S_i}\}_{i \in I}}

\inferrule{}{
    \dual{\Select\{l_i : S_i\}_{i \in I}} =
    \Branch \{l_i : \dual{S_i}\}_{i \in I}}

\inferrule{}{\dual{\End} = \End}
\end{mathpar}

\begin{mathpar}
\inferrule
    {\Gamma \types x : \End}
    {\Gamma \types \PO}
    \quad (\textsc{T-Inact})

\inferrule
    {\Gamma_1 \types P \\
     \Gamma_2 \types Q}
    {\Gamma_1 \circ \Gamma_2 \types \comp{P}{Q}}
    \quad (\textsc{T-Par})

\inferrule
    {\Gamma,x:T,y:\dual{T} \types P}
    {\Gamma \types \new{x}{y}P}
    \quad (\textsc{T-Res})

\inferrule
    {\Gamma_1 \types x:\Recv{T}S \\
     \Gamma_2,x:S,y:T \types P}
    {\Gamma_1 \circ \Gamma_2 \types \cin{x}{y}P}
    \quad (\textsc{T-In})

\inferrule
    {\Gamma_1 \types x:\Send{T}S \\
     \Gamma_2 \types v:T \\
     \Gamma_3,x:S \types P}
    {\Gamma_1 \circ \Gamma_2 \circ \Gamma_3 \types \cout{x}{v}P}
    \quad (\textsc{T-Out})

\inferrule
    {\Gamma_1 \types x:\Branch{\{l_i : S_i\}_{i \in I}} \\
     \Gamma_2,x:S_i \types P_i \\
     \forall i \in I}
    {\Gamma_1 \circ \Gamma_2 \types x \branch{\{l_i : P_i\}_{i \in I}}}
    \quad (\textsc{T-Branch})

\inferrule
    {\Gamma_1 \types x:\Select{\{l_i : S_i\}_{i \in I}} \\
     \Gamma_2,x:S_i \types P_i \\
     \exists j \in I}
    {\Gamma_1 \circ \Gamma_2 \types x \select{l_j}P}
    \quad (\textsc{T-Select})
\end{mathpar}

%%%%%%%%%%%%%%%%%%%%%%%%%%%%%%%%%%%%%%%%%%%%%%%%%%%%%%%%%%%%%%%%%%%
\section{The Coq proof assistant}\label{coq}
%%%%%%%%%%%%%%%%%%%%%%%%%%%%%%%%%%%%%%%%%%%%%%%%%%%%%%%%%%%%%%%%%%%

\todo{Powerful tactics}
\todo{Not so good with dependent types}
\todo{Equations package}

%%%%%%%%%%%%%%%%%%%%%%%%%%%%%%%%%%%%%%%%%%%%%%%%%%%%%%%%%%%%%%%%%%%
\section{Polymorphism}\label{polymorphism}
%%%%%%%%%%%%%%%%%%%%%%%%%%%%%%%%%%%%%%%%%%%%%%%%%%%%%%%%%%%%%%%%%%%

\cite{Wadler1989}

%%%%%%%%%%%%%%%%%%%%%%%%%%%%%%%%%%%%%%%%%%%%%%%%%%%%%%%%%%%%%%%%%%%
\section{Dependent types}\label{dependent-types}
%%%%%%%%%%%%%%%%%%%%%%%%%%%%%%%%%%%%%%%%%%%%%%%%%%%%%%%%%%%%%%%%%%%
\todo{Indexed datatypes}

%%%%%%%%%%%%%%%%%%%%%%%%%%%%%%%%%%%%%%%%%%%%%%%%%%%%%%%%%%%%%%%%%%%
\chapter{Encoding}\label{encoding}
%%%%%%%%%%%%%%%%%%%%%%%%%%%%%%%%%%%%%%%%%%%%%%%%%%%%%%%%%%%%%%%%%%%

%%%%%%%%%%%%%%%%%%%%%%%%%%%%%%%%%%%%%%%%%%%%%%%%%%%%%%%%%%%%%%%%%%%
\section{Parametric HOAS}\label{phoas}
%%%%%%%%%%%%%%%%%%%%%%%%%%%%%%%%%%%%%%%%%%%%%%%%%%%%%%%%%%%%%%%%%%%

\cite{Wadler1989}
\cite{Chlipala2008}
\todo{We use polymorphism to make channels opaque}
\todo{We do not do open processes}
\todo{We use polymorphism on messages to make processes traversable}

The introduction of both channels and received messages is modelled as function
abstraction in Coq, therefore \textbf{variables are handled transparently} -- no
substitution related lemmas are required. Channel types are parametrised to make
them opaque -- they cannot be illicitly created or inspected by the user.

%%%%%%%%%%%%%%%%%%%%%%%%%%%%%%%%%%%%%%%%%%%%%%%%%%%%%%%%%%%%%%%%%%%
\section{Continuation passing}\label{continuation-passing}
%%%%%%%%%%%%%%%%%%%%%%%%%%%%%%%%%%%%%%%%%%%%%%%%%%%%%%%%%%%%%%%%%%%

\cite{Vasconcelos2010}
\todo{We merge pi calculus and session types into one}

Assuming linearity, \textbf{processes are correct by construction}: the
processes that can be constructed depend on the session types of the channels in
the environment of the host language; an action strips off the outer layer of a
channel's session type -- modelling \textbf{continuation passing}.

%%%%%%%%%%%%%%%%%%%%%%%%%%%%%%%%%%%%%%%%%%%%%%%%%%%%%%%%%%%%%%%%%%%
\section{Linearity}\label{linearity}
%%%%%%%%%%%%%%%%%%%%%%%%%%%%%%%%%%%%%%%%%%%%%%%%%%%%%%%%%%%%%%%%%%%

\cite{Kobayashi1999}
\cite{Toninho2011}

%%%%%%%%%%%%%%%%%%%%%%%%%%%%%%%%%%%%%%%%%%%%%%%%%%%%%%%%%%%%%%%%%%%
\section{Type preservation}\label{type-preservation}
%%%%%%%%%%%%%%%%%%%%%%%%%%%%%%%%%%%%%%%%%%%%%%%%%%%%%%%%%%%%%%%%%%%

Ensuring that linearity is preserved through reduction is therefore essential:
\begin{theorem}
    $lin(P) \Rightarrow P \rightarrow Q \Rightarrow lin(Q).$
\end{theorem}

%%%%%%%%%%%%%%%%%%%%%%%%%%%%%%%%%%%%%%%%%%%%%%%%%%%%%%%%%%%%%%%%%%%
\chapter{Implementation}\label{implementation}
%%%%%%%%%%%%%%%%%%%%%%%%%%%%%%%%%%%%%%%%%%%%%%%%%%%%%%%%%%%%%%%%%%%

%%%%%%%%%%%%%%%%%%%%%%%%%%%%%%%%%%%%%%%%%%%%%%%%%%%%%%%%%%%%%%%%%%%
\section{Processes}\label{processes}
%%%%%%%%%%%%%%%%%%%%%%%%%%%%%%%%%%%%%%%%%%%%%%%%%%%%%%%%%%%%%%%%%%%
\subsection{Congruence}
\subsection{Reduction}

%%%%%%%%%%%%%%%%%%%%%%%%%%%%%%%%%%%%%%%%%%%%%%%%%%%%%%%%%%%%%%%%%%%
\section{Linearity check}\label{linearity-check}
%%%%%%%%%%%%%%%%%%%%%%%%%%%%%%%%%%%%%%%%%%%%%%%%%%%%%%%%%%%%%%%%%%%

%%%%%%%%%%%%%%%%%%%%%%%%%%%%%%%%%%%%%%%%%%%%%%%%%%%%%%%%%%%%%%%%%%%
\section{Linearity preservation}\label{linearity-preservation}
%%%%%%%%%%%%%%%%%%%%%%%%%%%%%%%%%%%%%%%%%%%%%%%%%%%%%%%%%%%%%%%%%%%

%%%%%%%%%%%%%%%%%%%%%%%%%%%%%%%%%%%%%%%%%%%%%%%%%%%%%%%%%%%%%%%%%%%
\chapter{Related work}\label{related-work}
%%%%%%%%%%%%%%%%%%%%%%%%%%%%%%%%%%%%%%%%%%%%%%%%%%%%%%%%%%%%%%%%%%%

%%%%%%%%%%%%%%%%%%%%%%%%%%%%%%%%%%%%%%%%%%%%%%%%%%%%%%%%%%%%%%%%%%%
\chapter{Conclusion}\label{conclusion}
%%%%%%%%%%%%%%%%%%%%%%%%%%%%%%%%%%%%%%%%%%%%%%%%%%%%%%%%%%%%%%%%%%%

%%%%%%%%%%%%%%%%%%%%%%%%%%%%%%%%%%%%%%%%%%%%%%%%%%%%%%%%%%%%%%%%%%%
\bibliographystyle{plain}
\bibliography{mproj}
\end{document}
