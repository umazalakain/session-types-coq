\documentclass{beamer}

\usepackage{mathtools}
\usepackage{xcolor}

\mode<presentation>
\setbeamertemplate{navigation symbols}{}
\setbeamertemplate{footline}{}

\title{
    Type-checking \\
    session typed $\pi$-calculus \\
    with Coq\\
}
\author{\textbf{Uma Zalakain}}
\institute{University of Glasgow}
\date{}

\begin{document}
\begin{frame}
    \titlepage
\end{frame}

\begin{frame}
    \frametitle{Background}
    \begin{itemize}
        \item Undergraduate degree at the University of Strathclyde\\
            {\small
            \textbf{Evidence-providing problem solvers in Agda} \\
            Supervised by \textit{Conor McBride} \\
            \scriptsize Solver for equations on monoids\\
            Incomplete solver for Presburger arithmetic \\
            }
        \item Master's degree at the University of Glasgow\\
            Supervised by \textit{Ornela Dardha} \\
            {\scriptsize
            First experience with session types and process calculi \\
            Started working in June \\
            Soon applying for PhD \\
            }
    \end{itemize}
\end{frame}

\begin{frame}
    \frametitle{Problem}
    \begin{block}{Formalising session typed $\pi$-calculus in Coq}
        \begin{itemize}
            \item subset (finite, no shared channels)
            \item strong correctness guarantees
            \item interesting modelling exercise
            \item perfect excuse to familiarise with Coq
        \end{itemize}
    \end{block}
\end{frame}

\begin{frame}
    \frametitle{Goal}
    \begin{block}{Correctness by construction}
        \begin{itemize}
            \item coq type-checks process $\iff$ process uses STs correctly
            \item bonus: the STs a process uses get type-inferred
        \end{itemize}
    \end{block}
\end{frame}

\begin{frame}
    \frametitle{Ingredients}
    \begin{block}{Continuation passing}
        \begin{itemize}
            \item $\xrightarrow{x : C[H.T]} f(x) \xrightarrow{y : C[T]} \ldots$
            \item channels type indexed by session types
        \end{itemize}
    \end{block}

    \begin{block}{HOAS}
        \begin{itemize}
            \item channels/messages as arguments
            \item variable references lifted to Coq
            \item no substitution lemmas
            \item no environments (only closed processes)
        \end{itemize}
    \end{block}

    \begin{block}{Parametric channel type}
        \begin{itemize}
            \item opaque unforgeable channels
        \end{itemize}
    \end{block}
\end{frame}

\begin{frame}
    \frametitle{Obstacle}
    \centering
    \LARGE
    $
    \xrightarrow{x : C[H.T]} f(x)
    \xrightarrow{y : C[T]} g(\colorbox{red}{x})
    \xrightarrow{y : C[T]} \ldots$
    \vfill
    i.e., linearity
\end{frame}

\begin{frame}
    \frametitle{Solution}
    \begin{itemize}
        \item define linearity as a inductive proposition on processes
        \item ensures session fidelity
        \item ensures session privacy
        \item ensures communication safety
        \item process is well-typed $\iff$ process is linear
    \end{itemize}
\end{frame}

\begin{frame}
    \frametitle{Type preservation}
    \LARGE
    \centering
    $\forall \ P \ Q : Process,$ 
    \vfill
    $\emptyset \vdash P, \,P \rightarrow Q \implies \emptyset \vdash Q$ \\
    $\Updownarrow$ \\
    $lin(P), \, P \rightarrow Q \implies lin(Q)$ \\
\end{frame}

\begin{frame}
    \frametitle{Recent explorations}
    \begin{itemize}
        \item keep an env with linearly available channels
        \begin{itemize}
            \item index processes with their env
            \item add created channels to env
            \item remove consumed channels from env
        \end{itemize}
        \item composition decides where channels go
        \item reduction by communication redirects channels through compositions
        \item \textcolor{violet}{processes are correct by construction}
        \item \textcolor{violet}{type preservation requires no proofs}
    \end{itemize}
\end{frame}

\begin{frame}
    \frametitle{Some questions}
    \begin{itemize}
        \item how to handle channels in the env?
        \begin{itemize}
            \item no decidable equality on session types
            \item destruct env list modulo some permutation?
            \item inductive $\in$ type with a \texttt{remove} operation?
        \end{itemize}
        \item is it possible to type-infer where channels go?
        \item will I bump into an invisible wall?
    \end{itemize}
\end{frame}

\end{document}
