% !TEX TS-program = XeLaTeX

\documentclass{beamer}

\usepackage{xcolor}
\usepackage{mathtools}
\usepackage{listings}

\usepackage{fontspec}
\usepackage{xunicode}
\usepackage{xltxtra}
\usepackage{babel}
\setmonofont{Latin Modern Mono}


\mode<presentation>
\setbeamertemplate{navigation symbols}{}
\setbeamertemplate{footline}{}
\usecolortheme[named=purple]{structure}

\title{
    Type-checking \\
    session-typed $\pi$-calculus \\
    with Coq\\
}
\author{Uma Zalakain}
\institute{University of Glasgow}
\date{}

\begin{document}
\begin{frame}
    \titlepage
\end{frame}

\begin{frame}
    \frametitle{Problem}
    \begin{block}{Formalising session typed $\pi$-calculus in Coq}
        \begin{itemize}
            \item subset (finite, no shared channels)
            \item strong correctness guarantees
            \item interesting modelling exercise
            \item perfect excuse to familiarise with Coq
        \end{itemize}
    \end{block}
\end{frame}

\begin{frame}
    \frametitle{Goal}
    \begin{block}{Correctness by construction}
        \begin{itemize}
            \item coq type-checks process $\iff$ process uses STs correctly
            \item bonus: the session types of channels are type-inferred
        \end{itemize}
    \end{block}
\end{frame}

\begin{frame}
    \frametitle{Ingredients}
    \begin{block}{Continuation passing}
        \begin{itemize}
            \item an action \texttt{A}\\
                consumes a channel \texttt{:A.T}\\
                creates a channel \texttt{:T}
        \end{itemize}
    \end{block}
\end{frame}

\begin{frame}
    \frametitle{Ingredients}
    \begin{block}{Abstraction}
        \begin{itemize}
            \item channels and messages as arguments
            \item variable references lifted to Coq
            \item no environments (only closed processes)
            \item no substitution lemmas
        \end{itemize}
    \end{block}
\end{frame}

\begin{frame}
    \frametitle{Ingredients}
    \begin{block}{Parametric channel type}
        \begin{itemize}
            \item opaque unforgeable channels
            \item indexed by session type
        \end{itemize}
    \end{block}
\end{frame}

\begin{frame}[fragile]
    \frametitle{Ingredients}
    \begin{verbatim}
| PNew
  : forall (s r : SType)
  , Duality s r
  → (Message C[s] → Message C[r] → Process)
  → Process

| PInput
  : forall {m : MType} {s : SType}
  , (Message m → Message C[s] → Process)
  → Message C[? m ; s]
  → Process
    \end{verbatim}

\end{frame}

\begin{frame}
    \frametitle{The catch}
    \centering
    \LARGE
    $
    \xrightarrow{x : C[A.T]} A(x)
    \xrightarrow{y : C[T]} A(\colorbox{red}{x})
    \xrightarrow{z : C[T]} \ldots$
\end{frame}

\begin{frame}
    \frametitle{Workaround}
    \begin{itemize}
        \item linearity as an inductive predicate on processes
        \item process traversal:
            \begin{itemize}
                \item need to construct messages of arbitrary type
                \item parametrise message types
                \item project messages types to the unit type
                \item cannot use constructs of the metalanguage anymore
            \end{itemize}
        \item process is linear $\iff$ process uses STs correctly
    \end{itemize}
\end{frame}

\begin{frame}
    \frametitle{Subject reduction}
    \LARGE
    \centering
    $\forall \ P \ Q : Process,$\\
    $P \rightarrow Q, \, lin(P) \implies lin(Q)$ \\
\end{frame}
\end{document}
